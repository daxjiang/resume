% !TEX program = xelatex

\documentclass{resume}
%\usepackage{zh_CN-Adobefonts_external} % Simplified Chinese Support using external fonts (./fonts/zh_CN-Adobe/)
%\usepackage{zh_CN-Adobefonts_internal} % Simplified Chinese Support using system fonts

\begin{document}
\pagenumbering{gobble} % suppress displaying page number

\name{Daxiong Jiang}

\basicInfo{
  \email{jiang.daxiong@outlook.com} \textperiodcentered\ 
  \phone{(+86) 18717822928} \textperiodcentered\ 
}

\section{\faGraduationCap\ Education}
\datedsubsection{\textbf{RWTH Aachen University}, Aachen, Germany}{2020 -- 2023}
\textit{M.S.} in Sustainable Energy Supply: Advanced Electrical Drive
\datedsubsection{\textbf{Tongji University}, Shanghai, China}{2016 -- 2020}
\textit{B.S.} in Building Electricity and Intelligence

\section{\faUsers\ Experience}
\datedsubsection{\textbf{NIO} Associate Engineer}{2023 -- Present}
\role{Labview, Python}{Brake System Endurance Test Bench}
Designed and deployed a NI PXI-based test system with servo motor control. Implemented QMH framework in NI PXI system, achieving High-precision synchronous data replay for CAN bus, wheel speed and motor control modules. Built a Windows-based test management application with:
\begin{itemize}
  \item File/data import, preprocessing, and post-processing capabilities
  \item Integration of pressure sensors and analog input card
  \item Real-time data visualization and customized data logging functions
\end{itemize}

\role{C++, Python}{\textbf{Adaptive Headlight Testing Platform}}
Contributed to the development of ADAS HIL and SIL test benches for multiple vehicle projects:
\begin{itemize}
  \item Developed middleware applications in adas-controller, enabling connection between external scenario simulator and software under test
  \item Modeled and adapted full-vehicle AD domain sensors (camera, lidar, radar, gnss, etc.) across different simulator to ensure proper perception functionality
  \item Independently developed LiDAR and Camera models in CarlaUE4 using Unreal Engine APIs
  \item Developed a high-speed (20 Mbps) UART communication parser on NI sbRIO FPGA for lighting systems
  \item Integrated various hardware/software components in the system and developed test-automation, significantly improving efficiency
\end{itemize}

\datedsubsection{\textbf{Leadrive GmbH} Intern}{2022 -- 2023}
\role{Labview, Simulink}{E-Motor HIL and Power-HIL Testbench}

\begin{itemize}
  \item Developed software using LabVIEW, focusing on data transfer between the host PC, real-time controller and FPGA
  \item Used Simulink to build the model of E-Motor, inverter, resolver, etc., and deployed the models on the FPGA.
  \item Designed current control algorithms for motor emulator
\end{itemize}

\section{\faCogs\ Skills}
\begin{itemize}[parsep=0.5ex]
  \item Programming Languages: Python, Labview, C++
  \item Platform: Linux/Windows
  \item Software: Matlab/Simulink, INCA, VTD, Carmaker, Veristand 
\end{itemize}

\section{\faHeartO\ Honors and Awards}
\datedline{DAAD Prize (Deutscher Akademischer Austauschdienst) }{2020}

\section{\faInfo\ Miscellaneous}
\begin{itemize}[parsep=0.5ex]
  \item Languages: Mandarin (Native), English (Fluent), German (Fluent)
\end{itemize}

%% Reference
%\newpage
%\bibliographystyle{IEEETran}
%\bibliography{mycite}
\end{document}
